\documentclass{bioinfo}
\copyrightyear{2009}
\pubyear{2009}

\usepackage{natbib}
\bibliographystyle{apalike}

\DeclareMathOperator*{\argmax}{argmax}

\begin{document}
\firstpage{1}

\title[Inference using sequencing data]{Using sequencing data for SNP calling, association test and estimate of population genetical parameters}

\author[Li]{Heng Li$^{1,}$\footnote{to whom correspondence should be addressed}}

\address{$^1$Broad Institute, 7 Cambridge Center, Cambridge, MA 02142, USA}

\history{Received on XXXXX; revised on XXXXX; accepted on XXXXX}
\editor{Associate Editor: XXXXXXX}
\maketitle

\begin{abstract}
\section{Motivation:}
Whereas with the rapidly decreasing sequencing cost it is now affordable to sequence
multiple human genomes, it would still be preferred to sequence many genomes at
relatively low coverge, typically 2--6 fold, to maximize the information about
human populations especially at the variants with low frequencies. Nonetheless, because the signal is weak in each low-coverage sample,
genotype ascertainment in each sample is usually inaccurate. The traditional
association test and the estimate of population parameters, which rely on accurate genotypes,
cannot be applied.
\section{Results:}
We present a statistical framework for calling SNPs,
performing association tests and inferring various population genetical parameters from
low-coverage sequencing data. Our method directly uses genotype likelihoods without
relying on accurate genotypes or linkage disequilibrium (LD). On real data, we showed that
the method has high accuracy in variant calling, and achieves a similar accuracy to imputation
based method on population parameter estimate, yet 60X faster and less affected by the lack of LD.
\section{Availability:} http://samtools.sourceforge.net
\section{Contact:} hengli@broadinstitute.org
\end{abstract}

\section{INTRODUCTION}

The 1000 Genomes Project sets an excellent example on how to design a sequencing project to
get the maximum output relating human populations. An important lesson from this project
is to sequence many human samples at relatively low coverage instead of
few samples at very high coverage. This is because with higher coverage,
we will merely reconfirm information from other reads, but with more samples, we
will greatly reduce the sampling fluctuations, gain power on variants present in
multiple samples and get access to many more rare variants. On the other hand, sequencing
errors counteract the power in variant calling, which necessitates a minimum coverage.
The optimal balancing point is broadly regarded to be in the 2--6 fold range,
depending on the sequencing error rate, level of linkage disequilibrium (LD) and the
aim of the project.

A major concern with this design is that at 2--6 fold coverage, 

\begin{methods}
\section{METHODS}
\subsection{Overview}

\begin{table}[!htb]
\processtable{Common notations}{\begin{tabular}{lp{7cm}}
\toprule
Symbol & Description \\
\midrule
$n$ & Number of samples \\
$m_i$ & Ploidy of the $i$-th sample ($1\le i\le n$)\\
$M$ & Total number of chromosomes in samples: $M=\sum_i m_i$\\
$d_i$ & Sequencing data (bases and qualities) for the $i$-th sample\\
$g_i$ & Genotype (the number of reference alleles) of the $i$-th sample \mbox{($0\le g_i\le m_i$)}$^1$\\
$\phi_k$ & Probability of observing $k$ reference alleles ($\sum_{k=0}^M\phi_k=1$) \\
$\Pr\{A\}$ & Probability of an event $A$\\
$\mathcal{L}_i(\theta)$ & Likelihood function for the $i$-th sample: $\mathcal{L}_i(\theta)=\Pr\{d_i|\theta\}$ \\
\botrule
\end{tabular}}{$^1$ In this article, we only consider biallelic variants.}
\end{table}

\subsection{Assumptions}

To derive the probabilistic framework, we have to make several assumptions.

\subsubsection{Site independency} We assume data at different sites are independent.
This is frequently not true in real data because sequencing and mapping are context
dependent. When there is an insertions or deletion (INDEL), sites nearby are also
correlated in alignment.

\subsubsection{Error independency and sample independency}
We assume that at a site the sequencing and mapping errors of different reads
are independent. As a result the likelihood functions of
different individuals are independent:
\begin{equation}
\mathcal{L}(\theta)=\prod_{i=1}^n\mathcal{L}_i(\theta)
\end{equation}
In real data, errors are frequently context dependent and the independency assumption
may not held. It is possible to model error dependency within an individual~\citep{Li:2008zr},
but the sample independency assumption is essential to all the derivations below.

\subsubsection{Biallelic variants}
We assume all variants are biallelic. In the human population, the fraction
of triallic SNPs is about 0.2\%~\citep{Hodgkinson:2010uq}. The biallele assumption
does not have a big impact to the modeling of SNPs.

\subsection{Computing genotype likelihoods}

For one sample at a site, the sequencing data $d$ is composed of
an array of bases on sequencing reads plus their qualities. As
we only consider biallelic variants, we may focus on the two most
evident types of nucleotides and drop the less evident types if present. Thus
at any site we see at most two types of nucleotides. This treatment
is not optimal, but mostly sufficient in practice.

Suppose at a site there are $k$ reads. Without losing generality, let the first $l$
bases ($l\le k$) be identical to the reference and the rest be
different. The error probability of the $j$-th read base is $\epsilon_j$.
Assuming error independency, we can derive that
\begin{equation}\label{equ:glk}
\mathcal{L}(g)=\frac{1}{m^k}\prod_{j=1}^l\Big[(m-g)\epsilon_j+g(1-\epsilon_j)\Big]\prod_{j=l+1}^k\Big[(m-g)(1-\epsilon_j)+g\epsilon_j\Big]
\end{equation}
where $m$ is the ploidy. For small $\{\epsilon_j\}$, the first-order approximation is
\begin{eqnarray*}
\mathcal{L}(g)&=&\left(1-\frac{g}{m}\right)^k\left(1-\frac{g}{m-g}\right)\left(\sum_{j=1}^l\epsilon_j-\sum_{j=l+1}^k\epsilon_j\right)\\
&&+\left(\frac{g}{m}\right)^l\left(1-\frac{g}{m}\right)^{k-l}+o(\epsilon^2)
\end{eqnarray*}
Intuitively, the first term attempts to explain
the deviation between $l/k$ and $g/m$ by sequencing errors, while
the second term attempts to explain the deviation by imperfect sampling.

%In particular, for a diploid sample ($m=2$), the likelihoods for $g=0,1,2$ are
%\begin{eqnarray*}
%\mathcal{L}(0)&=&\prod_{j=1}^l\epsilon_j\prod_{j=l+1}^k(1-\epsilon_j)\\
%\mathcal{L}(1)&=&\frac{1}{2^k}\\
%\mathcal{L}(2)&=&\prod_{j=1}^l(1-\epsilon_j)\prod_{j=l+1}^k\epsilon_j\\
%\end{eqnarray*}

\subsection{Maximum-likelihood inference}

\subsubsection{Estimating the site allele frequency}
In this section we estimate the reference allele frequency $\psi$.
For the $i$-th sample, let $m_i$ be the ploidy, $g_i$ 
the genotype and $d_i$ the sequencing data.
Assuming Hardy-Weinberg equilibrium (HWE), we can compute the likelihood of $\psi$:
\begin{equation}\label{equ:flk}
\mathcal{L}(\psi)=\sum_{g_1}\cdots\sum_{g_n}\prod_i\Pr\{d_i,g_i|\psi\}=\prod_{i=1}^n\sum_{g=0}^{m_i}\mathcal{L}_i(g)f(g;m_i,\psi)
\end{equation}
where $\mathcal{L}_i(g_i)$ is computed by Eq.~\eqref{equ:glk} and
\begin{equation}
f(g;m,\psi)=\binom{m}{g}\psi^g(1-\psi)^{m-g}
\end{equation}
is the probability mass function of the binomial distribution ${\rm Binom}(m,\psi)$.

Knowing the likelihood of $\psi$, we may directly find the max-likelihood estimate
numerically with, for example, Brent's method~\citep{Brent:1973kx}. An alternative approach is to 
infer with the estimation-maximization algorithm (EM), taking sample genotypes as missing data. Given
we know the estimate $\psi^{(t)}$ at the $t$-th iteration, the estimate at the $(t+1)$-th
iteration is
\begin{equation}
\psi^{(t+1)}=\frac{1}{M}\sum_{i=1}^n\frac{\sum_{g}g\mathcal{L}_i(g)f(g;m_i,\psi^{(t)})}{\sum_{g}\mathcal{L}_i(g)f(g;m_i,\psi^{(t)})}
\end{equation}
wheere $M=\sum_im_i$ is the total number of chromosomes in samples.

When the signal from the data is strong, or equivalently for each $i$, one of $\mathcal{L}_i(g)$
is much larger than others, the EM algorithm converges faster than the direct numerical
solution using Brent's method. However, when the signal from the data is weak,
Brent's method may converge faster than EM. In implementation, we apply 10 rounds of
EM iterations. If the estimate does not converge after 10 rounds, we switch to Brent's method.

\subsubsection{Estimating the genotype frequencies}
In this section, we assume all samples have the same ploidy: $m=m_1=\cdots=m_n$
and aim to estimate $\xi_g$, the frequency of genotype $g$. The likelihood
of $\{\xi_0,\ldots,\xi_m\}$ is:
\begin{equation}
\mathcal{L}(\xi_0,\ldots,\xi_m)=\prod_{i=1}^n\sum_{g=0}^{m}\mathcal{L}_i(g)\xi_g
\end{equation}
with the constraint $\sum_g\xi_g=1$. The EM iteration equation is
\begin{equation}
\xi^{(t+1)}_g=\frac{1}{n}\sum_{i=1}^n\frac{\mathcal{L}_i(g)\xi^{(t)}_g}{\sum_{g'}\mathcal{L}_i(g')\xi_{g'}^{(t)}}
\end{equation}

An important application of estimaing the genotype frequencies is to test HWE for diploid samples ($m=2$).
When genotypes are known, a common strategy is to take the frequency and perform a 1-degree $\chi^2$ test. This
approach would not work well for sequencing data as it does not account for the uncertainty in
genotypes especially when the average read depth of each individual is low. A more proper
method is to perform a likelihood-ratio test. The test statistics is
$$
D=-2\log\frac{\mathcal{L}(\hat{\psi})}{\mathcal{L}(\hat{\xi_0},\hat{\xi_1},\hat{\xi_2})}
$$
which approximately follows the 1-degree $\chi^2$ distribution.
It is easy to prove that when there is no uncertainty in data (i.e. for all $i$, there exists
$g$ such that $\mathcal{L}_i(g)=1$), the likelihood-ratio test
is reduced to the $\chi^2$ test using genotype data.

\subsubsection{Estimating haplotype frequencies between loci}
In this section, we assume all samples are diploid. Given $k$ loci,
let $\vec{h}=(h_1,\ldots,h_k)$ be a haplotype where $h_j$ equals 1 if
the allele at the $j$-th locus is identical to the reference, and equals 0 otherwise.
Let $\eta_{\vec{h}}$ be the frequency of haplotype $\vec{h}$ satisfying
$\sum_{\vec{h}}\eta_{\vec{h}}=1$, where
$$
\sum_{\vec{h}}\eta_{\vec{h}}=\sum_{h_1=0}^1\sum_{h_2=0}^1\cdots\sum_{h_k=0}^1\eta_{(h_1,\ldots,h_k)}
$$
Knowing the genotype likelihood at the $j$-th locus for the $i$-th individual $\mathcal{L}^{(j)}_i(g)$, we
can compute the haplotype frequencies iteratively with:
\begin{equation}\label{equ:hf}
\eta^{(t+1)}_{\vec{h}}=\frac{\eta_{\vec{h}}^{(t)}}{n}\sum_{i=1}^n\frac{\sum_{\vec{h}'}\eta_{\vec{h'}}^{(t)}\prod_{j=1}^k\mathcal{L}^{(j)}_i(h_j+h'_j)}
{\sum_{\vec{h}',\vec{h}''}\eta_{\vec{h'}}^{(t)}\eta_{\vec{h''}}^{(t)}\prod_{j}\mathcal{L}^{(j)}_i(h'_j+h''_j)}
\end{equation}
The time complexity of one iteration is $O(n\cdot 4^k)$ and thus
it is impractical to estimate the haplotype frequency for many loci jointly.
A typical use of Eq.~\eqref{equ:hf} is to measure linkage disequlibrium
(LD) between two loci using the $r^2$ or $D'$ statistics.

\subsubsection{Testing associations}
Suppose we divide samples into two groups of size $n_1$ and $n-n_1$, respectively, and
want to test if group 1 significantly differs from group 2. One possible test statistics
could be
\begin{equation}
D=-2\log\frac{\mathcal{L}(\hat{\psi})}{\mathcal{L}^{[1]}(\hat{\psi}^{[1]})\mathcal{L}^{[2]}(\hat{\psi}^{[2]})}
\end{equation}
where $\hat{\psi}$ is the max-likelihood estimate of the site allele frequency in all samples,
and $\hat{\psi}^{[1]}$ and $\hat{\psi}^{[2]}$ are the estimates of allele frequency in group 1 and group 2, respectively.
Under the null hypothesis, $D$ approximately follows the 1-degree $\chi^2$ distribution.
We can prove that when there is no uncertainty in genotypes, the $D$ statistics
is identical to the 1-degree $\chi^2$ statistics used in a genome-wide association study (GWAS).

\subsubsection{Estimating the allele count}
A major problem with Eq.~\eqref{equ:flk} is the assumption of HWE. This section
shows an alternative approach without this assumption.

For convenience, define random vector $\vec{G}=(G_1,\ldots,G_n)$ to be a genotype configuration.
and $X=\sum_iG_i$ to be the number of reference alleles in all the samples.
We have
$$
\Pr\{\vec{G}=\vec{g}|X=k\}=\delta_{k,s_n(\vec{g})}\prod_{i=1}^n\frac{\binom{m_i}{g_i}}{\binom{M}{k}}
$$
where $s_n(\vec{g})=\sum_i g_i$ is the total
number of reference alleles in a genotype configuration $\vec{g}$, and $\delta_{kl}$ equals 1 if $k=l$
and equals 0 otherwise. The likelihood of allele count is
\begin{equation}\label{equ:klk}
\mathcal{L}(k)=\Pr\{\vec{d}|X=k\}=\frac{1}{\binom{M}{k}}\sum_{g_1}\cdots\sum_{g_n}\delta_{k,s_n(\vec{g})}\prod_i\binom{m_i}{g_i}\mathcal{L}_i(g_i)
\end{equation}
where $\vec{d}=(d_1,\ldots,d_n)$ represents all sequencing data. To compute this probability efficiently, we define
$$
z_{jl}=\sum_{g_1=0}^{m_1}\cdots\sum_{g_j=0}^{m_j}\delta_{l,s_j(\vec{g})}\prod_{i=1}^j\binom{m_i}{g_i}\mathcal{L}_i(g_i)
$$
for $0\le l\le \sum_{i=1}^jm_i$ and $z_{jl}=0$ otherwise. $z_{jl}$
can be calculated iteratively with
\begin{equation}\label{equ:z}
z_{jl}=\sum_{g_j=0}^{m_j}z_{j-1,l-g_j}\cdot\binom{m_j}{g_j}\mathcal{L}_j(g_j)
\end{equation}
starting from $z_{00}=1$. Comparing the defintion of $z_{nk}$ and Eq.~\eqref{equ:z}, we know that
\begin{equation}\label{equ:klk2}
\mathcal{L}(k)=\frac{z_{nk}}{\binom{M}{k}}
\end{equation}
which computes the likelihood of the allele count.

Although the computation of the likelihood function $\mathcal{L}(k)$ is
more complex than of $\mathcal{L}(\psi)$, $\mathcal{L}(k)$ does not requires
HWE and can be computed for sample pools. It is a discrete function, which
is more convenient when we want to compute the posterior. In fact,
this section establishes the fundament of the Bayesian inference below.

\subsubsection{Numerical stability of the allele count estimate}
When computing $z_{jl}$ with Eq.~\eqref{equ:z}, floating point
underflow may occur given large $j$. A more numerically stable approach is to
compute $y_{jl}=z_{jl}/\binom{M_j}{l}$ instead, where
$M_j=\sum_{i=1}^j m_i$. Thus
\begin{equation}\label{equ:klky}
\mathcal{L}(k)=y_{nk}
\end{equation}
and $y_{jk}$ can be computed iteratively with
\begin{eqnarray}\label{equ:y}
y_{jk}&=&\left(\prod_{l=0}^{m_j-1}\frac{k-l}{M_{j}-l}\right)\sum_{g_j=0}^{m_j}y_{j-1,k-g_j}\cdot\binom{m_j}{g_j}\mathcal{L}_j(g_j)\\\nonumber
	&&\cdot\left(\prod_{l=g_j}^{m_j-1}\frac{M_{j-1}-k+l+1}{k-l}\right)
\end{eqnarray}
However, we note that $y_{jl}$ may decrease exponetially with increasing $j$. Floating point underflow
may still occur. An even better solution is to
rescale $y_{jl}$ for each $j$, similar to the treatment of the forward algorithm
for Hidden Markov Models~\citep{Durbin:1998uq}. In practical implementation,
we compute
\begin{equation}
\tilde{y}_{jl}=\frac{y_{jl}}{\prod_{j'=1}^jt_{j'}}
\end{equation}
where $t_{j}$ is chosen such that $\sum_{l}\tilde{y}_{jl}=1$.

As another implementation note, most $y_{jl}$ are close to zero and
thus $y_{nk}$ can be computed in a band rather than in a triangle. This
may dramatically speed up the computation of the likelihood.

\subsection{Bayesian inference}

\subsubsection{Calling variants}
In variant calling, we have a strong prior knowledge that at most of sites all samples only
have the reference alelle. We will show that this prior plays an important role
in reducing false positives.

To utilize the prior knowledge, we prefer a Bayesian inference for variant calling.
Let $\phi_k$, $k=1,\ldots,M$, be the probability of seeing $k$ reference alleles
among $M$ chromosomes/haplotypes. For convenience, define $\Phi=\{\phi_k\}$, which
is in fact the sample \emph{allele frequency spectrum} (AFS) for $M$ chromosomes. Recall
that $X$ is the number of reference alleles in sample. The posterior of $X$ is
$$
\Pr\{X=k|\vec{d},\Phi\}=\frac{\phi_k\Pr\{\vec{d}|X=k\}}{\sum_l\phi_l\Pr\{\vec{d}|X=l\}}
=\frac{\phi_k\mathcal{L}(k)}{\sum_l\phi_l\mathcal{L}(l)}
$$
where $\mathcal{L}(k)$ is defined by Eq.~\eqref{equ:klk} and computed by Eq.~\eqref{equ:klky}.
In variant calling, we define \emph{variant quality} as
$$
Q_{\rm var}=-10\log_{10}\Big(1-\Pr\{X=M|\vec{d},\Phi\}\Big)
$$
and call the site as a variant if $Q_{\rm var}$ is large enough. Because in deriving
$\mathcal{L}(k)$, we do not require the ploidy of each sample to be the same. The
variant calling method described here are in theory applicable to pooled resequencing with unequal pool size.

\subsubsection{Estimating sample allele frequency spectrum (AFS)}
For variant calling, we typically take the Wright-Fisher AFS as the prior. We can also
estimate the sample AFS with the maximum-likelihood inference when the Wright-Fisher prior
deviates from the data.

Suppose we have $L$ sites of interest and we want to estimate the frequency across these sites.
Let $X_a$, $a=1,\ldots,L$, be the reference allele count at site $a$. We can use an EM algorithm
to find $\Phi$ that maximizes $\Pr\{{\bf d}|\Phi\}$. The iteration equation is
$$
\phi_k^{(t+1)}=\frac{1}{L}\sum_a\Pr\{X_a=k|{\bf d},\Phi^{(t)}\}
$$
\subsubsection{Comparing two samples or two groups of samples}
To compare two groups of samples, a useful distribution is the joint posterior distribution
of allele counts in the two groups:
\begin{eqnarray*}
&&\Pr\{X^{[1]}=k^{[1]},X^{[2]}=k^{[2]}|d,\Phi\}\\
&=&\frac{\phi_{k^{[1]}+k^{[2]}}\mathcal{L}^{[1]}(k^{[1]})\mathcal{L}^{[2]}(k^{[2]})}{\sum_l\phi_l\mathcal{L}(l)}
\cdot\frac{\binom{M^{[1]}}{k^{[1]}}\binom{M^{[2]}}{k^{[2]}}}{\binom{M^{[1]}+M^{[1]}}{k^{[1]}+k^{[2]}}}
\end{eqnarray*}
With this posterior, we may compute the probability that a variant is
present in one sample but absent from the other, which is important to
the discovery of somatic mutations. It is also possible to derive association test
statistics based the joint distribution.

\end{methods}

\section{RESULTS}
\subsection{Implementation}
Most of equations for diploid samples ($m=2$) have been efficiently implemented
in the SAMtools software package, which is distributed under the MIT open source license,
free to both academic and commercial uses.

The SAMtools package consists of two key components {\tt samtools} and {\tt bcftools}.
The former computes the genotype likelihood $\mathcal{L}(g)$ using an improved version of
Eq.~\eqref{equ:glk} which considers error dependencies; the latter calls variants
and infers various statistics decribed in this article. To clearly separate the two steps,
we designed a new \emph{Binary variant call format} (BCF), which is the binary representation
of the variant call format (VCF) and is more compact and much faster to process than VCF.
On real data, computing genotype likelihoods especially for INDELs is typically 10 times slower than
variant calling. The separation of genotype likelihood computation and subsequent
inferences enhances the flexibility and improves the efficiency for inferring
AFS. {\tt Bcftools} also directly works with VCF files, but is less efficient.

\subsection{Calling SNPs for the 1000 Genomes pilot data}

\subsection{Estimating site allele frequnecies}
We downloaded European 

\section{DISCUSSIONS}

\section*{ACKNOWLEDGEMENTS}
\paragraph{Funding\textcolon}
\bibliography{samtools}

\end{document}
